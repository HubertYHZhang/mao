\documentclass[12pt]{article}

\usepackage{setspace}
\usepackage[margin=1in]{geometry}
\usepackage{xeCJK}
\usepackage{fontspec}
\setCJKmainfont[BoldFont={STHeiti}]{FandolSong}
\usepackage{indentfirst}

\onehalfspacing
\setlength{\parindent}{24pt}

\newcommand{\enabstractname}{Abstract}
\newcommand{\cnabstractname}{摘要}
\newenvironment{enabstract}{%
  \par
  \mbox{}\hfill{\bfseries \enabstractname}\hfill\mbox{}\par
  \vskip 2.5ex}{\par\vskip 2.5ex}
\newenvironment{cnabstract}{%
  \par
  \mbox{}\hfill{\bfseries \cnabstractname}\hfill\mbox{}\par
  \vskip 2.5ex}{\par\vskip 2.5ex}

\title{\textbf{邓小平“南方谈话”与当下的全面深化改革}}
\author{13\quad 张煜豪\quad 光华管理学院\quad 1900015810\quad email:1900015810@pku.edu.cn}
\date{}

\begin{document}

\maketitle

\begin{cnabstract}
    邓小平同志在1992年的南方谈话是极其重要的历史性谈话
\end{cnabstract}

\section{引言}
1992年,邓小平同志在武昌、深圳、珠海、上海等地发表了一系列的谈话,史称“南方谈话”。其时,中国正处在改革的停滞期,1989年以来的政治风波使得经济和政治改革的进程受到了严重的阻碍,经济发展的动力也在逐渐消失。许多人对改革进行下去的信心产生了动摇和怀疑。邓小平同志在南方谈话中,对中国的改革开放进程进行了全面的总结,提出了一系列的改革措施,为中国的改革开放进程注入了新的动力,使得中国的改革开放进程重新回到了正轨上来。本文将对邓小平同志的南方谈话进行全面的解读,分析其对当下全面深化改革的启示。总的来说,小平同志的讲话启示我们,要敢于改革,破除体制机制和思想上的束缚,才能取得生产力的大发展,才能全面建成社会主义现代化强国。小平同志的讲话还启示当代的改革者,要走出一条中国特色的现代化道路,构建“制度优势”,方能立于不败之地。

实践是检验真理的唯一标准。

\section{历史背景}
1978年的十一届三中全会正式拉开了改革开放的序幕,随之而来的是农村和城市在经济方面的大变化:农村实行了家庭联产承包责任制,大幅提高了农民的生产积极性和生产效率,1978-1984年期间,农业的年增长率一度达到了8.8\%左右,是年经济增长率的3倍。\footnote{《伟大的中国经济转型》p405}在政策开放后,乡镇集体企业也成为了经济发展的重要力量。而城市经济在经历一系列针对国有企业的改革后也表现出了活力。但与此同时,价格机制的引入也对人们的生活有所冲击,由于生产能力的缺乏和宏观经济的波动?,物价不可避免地出现上涨的趋势。物价“双轨制”下出现的各种倒卖、腐败现象也引发了社会上的不满。来自西方的各种思潮也影响着人们的价值观和情绪。各种矛盾激化的结果是80年代末的政治风波,它使得改革不得不被搁置以稳定局面,国家一度陷入“改革要不要继续下去”的争论当中。

然而,不继续改革开放,经济将要陷入停滞乃至倒退。在这种情况下,原本已经退出政治舞台的邓小平同志再次站了出来,在武昌、深圳、珠海、上海等改革开放的前沿发表了一系列的谈话,为中国的改革开放进程注入了新的动力。

\section{“南方谈话”的主要内容}
邓小平同志在“南方谈话”中讨论了关于中国发展前景的一系列问题,并指出了他对于这些问题的看法和路线指引。下面大致总结一下这些谈话的主要内容:

一、改革开放是经济发展、人民生活富裕幸福的保障,虽然“六四”动摇了部分人改革的决心,但恰是前期的改革开放使得人们对国家有了更多的信心,才能够帮助我们度过这一难关。

二、改革开放就要敢作敢为,不要怕所谓“社会主义”和“资本主义”的争论,计划和市场并不是决定社会制度和意识形态的标准。只要有利于生产力发展,有利于消除贫困、消除两极分化、消灭剥削、共同富裕的,就是社会主义的。不能想着一开始就把所有的事情做对,资本主义使用的工具,在调查研究之后可以大胆去尝试,错了及时纠正即可。“必须大胆吸收和借鉴人类社会创造的一切文明成果,吸收和借鉴当今世界各国包括资本主义发达国家的一切反映现代社会化生产规律的先进经营方式、管理方法。”\footnote{《邓小平文选》第三卷p373}当前(指当时)主要的问题是防止“左”,防止搞过于激进的社会主义,重蹈覆辙。

三、未来的改革开放要努力让发展水平上一个新台阶。科学技术是第一生产力,要搞好科学和教育,关注科学家。

四、坚持“两手抓”。一手抓改革开放,一手要打击各种违法犯罪活动。要做好反腐工作,做好法制建设,并且用法制建设做好廉政建设;坚持人民民主专政,防范资产阶级自由化。总的来说,经济建设和政治治理两方面都要搞好。

五、在党的工作方面,要做好组织建设,培养“革命化、年轻化、知识化、专业化”的领导班子,健全人才培养机制和老干部的退出机制,防止“老人政治”。警惕形式主义和官僚主义。坚持基本路线,防范资产阶级自由化。只有把共产党内部搞好了,中国才能发展好。

五、在实践中学习马克思主义,贯彻马克思主义。改革开放靠的不是本本主义,而是靠的在实践中摸索(也即我们常说的“摸着石头过河”)尊重人民群众的实践智慧,效果好的经验可以尝试推广。“马克思主义是很朴实的东西,很朴实的道理。”\footnote{《邓小平文选》第三卷p382}

六、社会主义取代资本主义是历史发展的必然规律,我们要坚持对发展社会主义的信心,将中国发展成社会主义强国(建国一百年成为中等发达国家),成为反对霸权主义、维护世界和平与发展的重要力量。

\section{新时代的全面深化改革与三十年前的“南方谈话”}
回顾邓小平同志三十年前的讲话,依然感觉振聋发聩。这些讲话几乎全面地概括总结了我国在改革开放中遇到的问题和未来依然会遇到的问题,并提出了基本的路线和方法指引。

\subsection{解放和发展生产力}
进入新时代后,我国社会的主要矛盾转变为人民群众日益增长的对美好生活的需要和不平衡不充分的发展之间的矛盾。然而,这一转变并不妨碍小平同志的观点对当下发展的指导作用。

首先,要解决不平衡不充分发展的问题,依然要靠不断解放和发展生产力。生产力发展不单单是数量上的增加,更是质量上的提升,而质量上的提升正是解决不平衡与不充分所需要的。过去十年,我国深入进行“供给侧结构性改革”,进行产业转型和升级,大力发展服务业,优化产业结构,大力促进了生产力的质量提升。



科学技术是第一生产力,江泽民同志提出“科教兴国”,习近平总书记提出“创新驱动发展战略”,将创新摆在所有工作的核心位置,无疑都是对这一论断的继承和发展。

\subsection{国家治理现代化}
在经济发展的过程中,随着逐利机会的涌现,不免会出现许多游走在法律规范边缘乃至违反法律的行为,这种现象发生在中国共产党的干部和政府官员身上,危害尤其严重,是对人民群众利益的损害、对公平法治的无视、对社会秩序的挑衅、对人民政权的合法性的伤害。因此,小平同志在经历了“六四”的风波过后,敏锐地指出了这一点,提出了“两手抓”,不仅要抓普通的违法犯罪,更要抓内部的腐败,做好法制建设。

十八大以来,党中央开展了一系列反腐行动,进行党风廉政建设。据统计,截至2022年4月,全国纪检监察机关共立案审查调查438.8万件、470.9万人\footnote{新华社2022年9月6日报道},反腐力度空前。2018年,国家监察委员会成立,意味着中国共产党的廉政建设法制化程度迈上了新台阶。小平同志对我国政治治理改革建设的期望,终于在二十多年后得到更加充分的实现。

跟随邓小平同志的指引,我们更加清晰认识到政治体制改革与经济发展之间的紧密关联。亲清型政商关系。服务型政府。对产权的保护,对市场公平竞争的维护,坚决维护广大人民群众而不是少数权贵的利益。

\subsection{马克思主义的中国化}


\subsection{国际与国内的关系}


\section{反思与补充}
马克思主义中国化是一个不断自我完善、自我扬弃的过程。站在新时代的角度上,我们应当对邓小平同志的谈话有更新的认识,并基于邓小平理论形成更完善的一套关于社会主义现代化的理论认知。

首先,关于什么是社会主义,如何通过借鉴资本主义社会的工具来完善社会主义的论断,通常被总结为“不管白猫黑猫,能捉老鼠的就是好猫”。\footnote{《邓小平文选》第一卷}

市场制度的改革,并没有指出更加明确的方向。当时改革还没有进入深水区,经济增长依然能够通过对价格机制的放开、对国有企业的放活、大力进行投资建设来实现,而当下更加需要市场和政治制度的完善来保障市场在资源配置中起决定性作用,实现高质量发展和持续的经济增长。




\end{document}